\startconclusionpage
В настоящей работе был описан подход, позволяющий преобразовать
пользователей какого-либо онлайн-сервиса в компактные
численные вектора пригодные для использования алгоритмами машинного обучения,
используя информацию о музыкальных исполнителях, которых прослушивают
пользователи.

Стоит отметить, что описанный подход вычисления матрицы
<<исполнитель-пользователь>> на основе порядка следования исполнителей (см. 
раздел~\ref{ssec:artist_order}) может быть использован для
любых текстовых данных, где присутствует признак ранжирования
терминов в документах. Остальные вариации предложенной схемы
преобразования данных (см. раздел~\ref{sec:general_approach})
могут быть использованы для любых данных, вид которых схож с
текстовыми данными.

С одной стороны, универсальность предложенного подхода является также
и его недостатком. Недостаток заключается в том, что признаковое
описание документов, вычисленное с применением данного подхода,
не основывается на каких-либо специфических предположениях, которые
могут иметь место в конкретной задаче. С другой стороны, в каждой
конкретной задаче можно предложить способ вычисления матрицы
<<термин-документ>>, основываясь на специфике задачи, а затем
применить метод построения признакового описания документа на
основе техники векторного представления слов (см. 
раздел~\ref{ssec:docs_word_embedding}).

В рамках исследования был проведён эксперимент в котором решалась
задача определения пола и возраста пользователей сайта Last.fm с
использованием предложенного подхода. Результаты эксперимента
показали, что наилучшего результата позволяет получить вариация
подхода, в которой матрица <<исполнитель-пользователь>> вычисляется
с использованием формулы log-entropy (см. формулу~\ref{eq:log_entropy}),
а матрица <<пользователь-признак>>~--- на основе техники векторного
представления слов. Полученный результат превосходит результат,
достигнутый ранее в работе~\cite{wu2014gender}, а также результат,
полученный при помощи стандартного подхода на основе латентного
семантического анализа.

Для проверки утверждения о том, что предложенный подход применим и в других
задачах, он был апробирован на задаче \textit{Bag of Words Meets Bags
of Popcorn}\footnote{https://www.kaggle.com/c/word2vec-nlp-tutorial}
с сайта \textit{Kaggle.com}\footnote{https://www.kaggle.com},
которая является задачей бинарной классификации текстовых документов.
Обучающая выборка включает в себя как документы, у которых
присутствуют метки классов, так и те, у которых они отсутствуют.
Для построения модели использовались только документы обучающей
выборки, у которых имеются метки классов. Матрица <<термин-документ>>
вычислялась на основе формулы log-entropy. Алгоритм LSI использовался
с параметром \textit{num\_topics} равным 300. Алгоритм Word2Vec
использовался с параметрами \textit{size} = 300 и \textit{window} =
10. В качестве алгоритма классификации был использован алгоритм
\textit{LogisticRegressionClassifier} с 
параметром \textit{class\_weight} = \textit{auto} из библиотеки \textit{gensim}.
Термины из документов извлекались с помощью программной библиотеки
\textit{nltk}~\cite{nltk}. Для модели на основе LSI достигнутый
результат равен 0{,}86660, для модели на основе Word2Vec~---
0{,}88048. Результат является площадью под
ROC-кривой. Как видно, предложенный подход показал
приемлемый результат и для данной задачи

Развитием данного исследование может послужить апробирование 
предложенного подхода на других наборах данных, в других задачах,
а также с использованием других алгоритмов латентного семантического
анализа и векторного представления слов.

Кроме того, возможно, имеет место улучшение подхода к преобразованию
документов в векторное представление на основе линейной комбинации
векторов, полученных с помощью word embedding, таким образом, чтобы
настраивать коэффициенты перед векторами каким-либо способом, вместо
использования значений из матрицы <<термин-документ>>.

