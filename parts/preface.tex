\startprefacepage

Благодаря стремительному развитию компьютерных технологий, глобальная
сеть Интернет в современное время играет большую роль. Миллионы
людей по всему миру пользуются социальными сетями, интернет-магазинами
и другими онлайн-сервисами. Крупнейшие сайты достигают крайне
впечатляющих показателей посещаемости. Например, по некоторым данным 
сайт \textit{Facebook}\footnote{https://facebook.com}
является крупнейшей социальной сетью в мире, которой пользуются
примерно 1.49 миллиардов пользователей\footnote{http://www.statista.com/statistics/265831/number-of-unique-us-visitors-to-facebooykcom/},
а русскоязычную социальную сеть \textit{ВКонтакте}\footnote{https://vk.com}
посещают более 80 миллионов пользователей в 
день\footnote{https://vk.com/page-47200925\_44240810}. Кроме того,
пользователи оставляют огромное количество персональной информации 
в открытом доступе. По официальным данным пользователи сервиса 
\textit{Instagram}\footnote{https://www.instagram.com/} ежедневно
публикуют более 80 миллионов 
фотографий\footnote{https://www.instagram.com/press/}, а
в популярном сервисе микроблогов \textit{Twitter}\footnote{https://twitter.com}
пользователи пишут примерно 6000 сообщений
ежесекундно\footnote{http://www.internetlivestats.com/twitter-statistics/}.
Стоит также отметить, что многие онлайн-сервисы дают
возможность пользователям заполнять личные анкеты,
в которых, как правило, указана такая информация как
пол, дата рождения, город, хобби и так далее.

Наличие такого большого количества информации предоставляет возможность
анализировать эти данные. В частности, методы машинного обучения 
позволяют восстанавливать неизвестные характеристики пользователей по имеющимся.
В качестве неизвестных характеристик могут выступать как те,
что отсутствуют лишь у части пользователей, так и те, что недоступны
вообще. Во втором случае возможно использование эмпирических
предположений или информации, полученной из других источников.

Задача восстановления неизвестных характеристик пользователей 
(задача профилирования пользователей) не просто представляет
научный интерес, а она также имеет практическое применение. 
Например, такие характеристики, как пол и возраст, являются
одними из ключевых признаков, использующихся в рекомендательных
системах~\cite{swearingen2001beyond,adomavicius2005toward}.
Следовательно устранение неполноты в данных с приемлемой точностью
может помочь улучшить системы рекомендаций.

В настоящем исследовании представлены подходы, позволяющие
строить модели векторного представления данных для некоторого класса
задач определения характеристик пользователей. Состоятельность
предложенных подходов продемонстрирована на примере задачи определения
демографических характеристик пользователей на основе анализа их
музыкальных интересов. Музыка, которую слушают пользователи, является
достаточно специфичной информацией, поэтому она используется крайне редко.
Ввиду сказанного, предполагается, что предложенный подход может помочь
улучшить существующие алгоритмы восстановления характеристик пользователей,
которые не используют информацию о музыке.
