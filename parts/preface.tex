\startprefacepage
В~\cite{swearingen2001beyond,
adomavicius2005toward,
burger2006exploration,
peersman2011predicting,
turdakov2013opredelenie,
schwartz2013personality,
rosenthal2011age,
farseev2015harvesting,
christenson1988genre,
liu2012inferring,
wu2014gender,
deerwester1990indexing,
mikolov2013efficient,
suykens1999least,
hsu2003practical} наше время Интернет играет большую роль. Миллионы людей
по всему миру пользуются социальными сетями, интернет-магазинами
и другими онлайн-сервисами. Пользователи оставляют о себе
огромное количество информации в открытом доступе даже 
не замечая этого. Благодаря этому появляется возможность
определять <<скрытые>> характеристики пользователей на основе
имеющейся информации. Такими характеристиками, например, 
являются пол и возраст. Многие сайты предоставляют возможность 
указывать эти параметры, но часто это не является обязательным
требованием. Отсюда возникает задача устранения неполноты в 
данных. Восстановление недостающих характеристик пользователей 
позволяет улучшить различные системы
рекомендаций~\cite{recommender2001,recommender2005}.

Существует множество исследований, показывающих возможность 
определять пол и возраст на основе косвенных признаков.
В работе~\cite{burger2006} исследовалась зависимость 
между текстом блогеров и их возрастом.
Авторы работ~\cite{peersman2011,turdakov2013,schwartz2013} 
предлагают подходы к определению демографических характеристик 
на основе сообщений пользователей. Текст~--- не единственная
информация, которая может быть использована. Например, в
работе~\cite{rosenthal2011} учитывалось также
поведение пользователей. А в работе~\cite{farseev2015}
для определения демографических характеристик использовались
также фотографии и геолокация пользователей.

Использование информации о музыке, которую слушают пользователи, 
также возможно. В статье~\cite{christenson1988} была показана
корреляция между предпочтением определённых жанров и полом у
группы студентов. В работе~\cite{liu2012} пол и возраст определялся
по истории прослушиваний музыкальных композиций.
В исследовании~\cite{wu2014} используется 50 наиболее
прослушиваемых композиций каждого пользователя сайта Last.fm.

В рамках настоящей работы предложены подходы к определению
демографических характеристик пользователей на данных, которые
использовались в последней упомянутой статье. Предполагается,
что описанные методы могут помочь улучшить существующие алгоритмы,
не использующие информацию о музыке пользователей.

Задача определения пола решалась как задача 
классификации, а задача определения возраста~--- как
задача восстановления регрессии.
