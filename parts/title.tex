%%% Общая информация
\author{Семёнов Артём Сергеевич}{Семёнов А. С.}
\publishyear{2016}
\studygroup{M4238}
\supervisor{Фильченков Андрей Александрович}{Фильченков А. А.}{канд. физ.-мат. наук}{Университет ИТМО}
\title{Определение демографических характеристик пользователей социальных
       сетей на основе анализа их музыкальных интересов}

%%% Задание
\technicalspec{
    Основной целью исследования является разработка подхода к решению задачи определения
    демографических характеристик пользователей социальных сетей на основе анализа их
    музыкальных интересов, а далее проведение экспериментов с использованием подхода
    на реальных данных для подтверждения состоятельности предложенного подхода.
    В качестве данных для эксперимента предлагается взять данные,
    которые использовались в каком-либо существующем исследовании.
}
\plannedcontents{
    Пояснительная записка должна содержать обзор существующих исследований,
    в которых решалась задача профилирования пользователей социальных сетей.
    В пояснительной записке также должено присутствовать подробное описание
    предлагаемого подхода к решению задачи. Наконец, проведённый
    в рамках исследования эксперимент должен быть детально описан с указанимем
    использованных программных средств и полученных результатов.
}
\plannedgraphics{
    Графические материалы и чертежи отсутствуют.
}
\plannedsources{
    \begin{enumerate}
        \item {Christenson P. G., Peterson J. B. Genre and gender in the
               structure of music preferences // Communication
               Research. — 1988. — Т. 15, No 3. — С. 282—301}
        \item {Liu J.-Y., Yang Y.-H. Inferring personal traits from music listening
               history // Proceedings of the second international ACM workshop on Music
               information retrieval with user-centered and multimodal
               strategies. — 2012 — С 31—36}
        \item {Wu M.-J., Jang J.-S. R., Lu C.-H. Gender Identification
               and Age Estimation of Users Based on Music 
               Metadata. // ISMIR. — 2014. — С. 555—560}
    \end{enumerate}
}
\addstage{Обзор существующих исследований по теме}{12.2014}
\addstage{Постановка и формализация задачи}{03.2015}
\addstage{Обзор литературы по требуемым методам}{05.2015}
\addstage{Разработка метода решения поставленной задачи}{11.2015}
\addstage{Проведение экспериментов}{03.2016}
\addstage{Написание текста магистерской диссертации}{04.2016}
\addstage{Окончательные правки по тексту магистерской диссертации}{05.2016}

%%% Аннотация
\researchaim{
    Целью настоящего исследования является разработка методологии построения
    пространства признакового описания пользователей социальных сетей на
    основе анализа их музыкальных интересов с целью определения
    <<скрытых>> демографических характеристик пользователей.
}
\researchtargets{
    \begin{enumerate}
        \item {Постановка задачи определения демографических характеристик
               пользователей по их музыкальным предпочтениям}
        \item {Разработка подхода к решению поставленной задачи}
        \item {Проведение экспериментов, результаты которых демонстрируют
               тот факт, что предложенный подход является состоятельным}
    \end{enumerate}
}
\advancedtechnologyusage{
    В рамках исследования для проведения экспериментов были использованы
    следующие программные средства: язык программирования Python, библиотеки
    numpy, scipy, pandas, scikit-learn, gensim, BigARTM.
}
\researchsummary{
    В рамках исследования был предложен подход определения пола и возраста
    пользователей сайта Last.fm на основе названий наиболее прослушиваемых
    ими музыкальных исполнителей. Подход основан на таких техниках и
    концепциях как латентный семантический анализ, векторное представление
    слов, матрица <<термин-документ>>. Достигнутые показатели: точность
    определения пола~--- 83.86\%, средняя абсолютная ошибка определения
    возраста~--- 2.65\%. Результаты, полученные в рамках данного
    исследования, несколько превосходят результаты, которые были достигнуты
    ранее.
}
\researchfunding{
    Гранты, ага.
}
\researchpublications{
    \begin{refsection}
    На тему диссертации имеются публикации: 
    \nocite{semenov1,semenov2,semenov3}
    \printannobibliography
    \end{refsection}
}

\maketitle{Магистр}

