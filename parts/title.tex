\studygroup{M4238}
\title{Определение демографических характеристик пользователей социальных
       сетей на основе анализа их музыкальных интересов}
\titleannotation{\uline{Определение демографических характеристик пользователей социальных
                 сетей на основе анализа их музыкальных интересов\hfill~}}
\author{Семёнов А.С.}
\supervisor{Фильченков А.А.}
\supervisordegree{канд. физ.-мат. наук, доцент}
\publishyear{2016}

\researchdirections{
    Целью настоящего исследования является разработка методологии построения
    пространств признакового описания пользователей социальных сетей на
    основе анализа их музыкальных интересов с целью определения
    <<скрытых>> демографических характеристик пользователей.
}

\researchpart{
    В рамках данной работы предложен подход, позволяющий описать
    пользователей конкретной социальной сети векторным пространством,
    основываясь на названиях музыкальных исполнителей, прослушиваемых
    пользователями. В основе предложенного подхода лежит использование методов и
    концепций, часто применяемых в задачах анализа текста: 
    матрица <<термин-документ>>, латентный семантический анализ 
    (тематическое моделирование), векторное представление слов (word embedding).
    Кроме того, описанный подход применим к более широкому кругу задач.
    Состоятельность модели продемонстрирована на примере задачи определения
    пола и возраста пользователей сайта Last.fm. Достигнутые показатели:
    точность определения пола~--- 82.46\%, средняя абсолютная ошибка
    определения возраста~--- 3.38. Результаты, полученные в рамках
    данного исследования, несколько превосходят результаты,
    которые были достигнуты ранее.
}

\economicpart{
    Данная работа не предполагает извлечения прямой экономической выгоды 
    из полученных результатов.
}

\ecologypart{
    Результатом работы является исследование, которое ни коим образом 
    не касается экологической безопасности.
}

\novelty{
    В рамках настоящего исследования описан подход, позволяющий
    представить пользователей компактными численными векторами,
    основываясь на данных, похожих на текст, но не являющихся
    текстом в традиционном смысле. Предложенный подход представляет
    из себя способ построения обобщённой модели со множеством вариаций, 
    не использующийся в таком виде ранее.
}

\cwpublications{
    Работа является продолжением исследования... % TODO
}

\practicalimplications{
    Демографические характеристики пользователей являются
    ключевыми признаками, которые используются при реализации
    рекомендательных систем. Таким образом, описанный подход
    призван улучшить рекомендательные системы путём
    устранения неполноты в данных о пользователях. Кроме
    того, предложенный подход определения демографических
    характеристик пользователей на основе музыкальных
    предпочтений может помочь улучшить существующие алгоритмы,
    которые не используют информацию о музыке пользователей.
}

\makemastertitle
